\documentclass[12pt, letterpaper]{article}
\usepackage[left=2.5cm,right=2.5cm, top=2.5cm, bottom=2.5cm]{geometry}
\usepackage{fancyhdr}
\pagestyle{fancy}
\fancyhead[R]{Flaherty, \thepage}
\renewcommand{\headrulewidth}{2pt}
\setlength{\headheight}{15pt}
\usepackage{lipsum}
\usepackage{amsmath}
\usepackage[makeroom]{cancel}
\usepackage{cancel}
\usepackage{array,polynom}
\newcolumntype{C}{>{{}}c<{{}}} % for '+' and '-' symbols
\newcolumntype{R}{>{\displaystyle}r} % automatic display-style math mode 
\usepackage{xcolor}
\newcommand\Ccancel[2][black]{\renewcommand\CancelColor{\color{#1}}\cancel{#2}}
% Define a custom environment for examples with an indent

\newenvironment{ex}{
	\par\smallskip % Add some vertical space before the example
	\noindent\textit{Example:\hspace{-0.25em}}
	\leftskip=0.5em % Set the left indent to 1em (adjust as needed)
}{
	\par\smallskip % Add some vertical space after the example
	\leftskip=0em % Reset the left indent
}

\newenvironment{nonex}{
	\par\smallskip % Add some vertical space before the example
	\noindent\textit{Non-example:\hspace{-0.25em}}
	\leftskip=0.5em % Set the left indent to 1em (adjust as needed)
}{
	\par\smallskip % Add some vertical space after the example
	\leftskip=0em % Reset the left indent
}
\newcommand{\mymatrix}[1]{
	\renewcommand{\arraystretch}{0.5} % Adjust vertical spacing%
	\setlength\arraycolsep{3pt}       % Adjust horizontal spacing%
	\scalebox{0.90}{                  % Change font size%
		$\begin{bmatrix}
			#1
		\end{bmatrix}$
	}                   
	\renewcommand{\arraystretch}{1.0} % Reset vertical spacing
	\setlength\arraycolsep{6pt}       %Adjust horizontal spacing%
}

\usepackage{amssymb}
\usepackage{bbm}
\usepackage{mathrsfs}
\usepackage[toc]{glossaries}
\usepackage{amsthm}
\usepackage{indentfirst}
\usepackage[utf8]{inputenc}
\usepackage[thinc]{esdiff}
\usepackage{graphicx}
\graphicspath{{./images/}}
\usepackage{subfig}
\usepackage{chngcntr}
\usepackage{placeins}
\usepackage{caption}
\usepackage{float}
\usepackage{comment}
\usepackage{sectsty}
\sectionfont{\fontsize{15}{15}\selectfont}
\usepackage{subcaption}
\setlength\abovedisplayskip{0pt}
\usepackage[hidelinks]{hyperref}
\usepackage[nottoc,numbib]{tocbibind}
\renewcommand{\qedsymbol}{\rule{0.7em}{0.7em}}
\newcommand{\Mod}[1]{\ (\mathrm{mod}\ #1)}
\counterwithin{figure}{section}
\usepackage{centernot}
\usepackage{enumitem}
\theoremstyle{definition}
\newtheorem{exmp}{Example}
\newtheorem{nonexmp}{Non-Example}
\newtheorem{theorem}{Theorem}[section]
\newtheorem{corollary}{Corollary}[theorem]
\newtheorem{definition}{Definition}[section]
\newtheorem{lemma}{Lemma}[theorem]
\numberwithin{equation}{section}
\newcommand{\mydef}[1]{(Definition \ref{#1}, Page \pageref{#1})}
\newcommand{\mytheorem}[1]{(Theorem \ref{#1}, Page \pageref{#1})}
\newcommand{\mylemma}[1]{(Lemma \ref{#1}, Page \pageref{#1})}
\newcommand{\clickableword}[2]{\hyperref[#1]{#2}}

%underscript for operations%
\newcommand{\+}[1]{+_{\scalebox{.375}{#1}}}
\newcommand{\mult}[1]{\cdot_{\scalebox{.375}{#1}}}

%blackboard for letters%
\newcommand{\E}{\mathbb{E}}
\newcommand{\V}{\mathbb{V}}
\newcommand{\R}{\mathbb{R}}
\newcommand{\N}{\mathbb{N}}
\newcommand{\C}{\mathbb{C}}
\newcommand{\Z}{\mathbb{Z}}
\newcommand{\F}{\mathbb{F}}
\newcommand{\K}{\mathbb{K}}
\newcommand{\1}{\mathbbm{1}}

\title{Time Series HW \# 3}
\author{Liam Flaherty}
\date{\parbox{\linewidth}{\centering%
		Professor Martin\endgraf\bigskip
		NCSU: ST546-001\endgraf\bigskip
		October 2, 2024 \endgraf}}

\begin{document}
\maketitle
\thispagestyle{empty}

\newpage\clearpage\noindent


\noindent\textbf{1) \boldmath{Consider the following models: \vspace{\baselineskip}
\begin{align*}
	&A: (1-B)\tilde{Z}_t=(1-1.5B)a_t\\
	&B: (1-0.8B)\tilde{Z}_t=(1-0.5B)a_t\\
	&C: (1-1.1B+0.8B^2)\tilde{Z}_t=(1-1.7B+0.72B^2)a_t\\
	&D: (1-0.6B)\tilde{Z}_t=(1-1.2B+0.2B^2)a_t
\end{align*}}}

\vspace{\baselineskip}
\noindent\textbf{\boldmath{a. Verify whether or not the model for $Z_t$ is stationary and/or invertible.}}
\vspace{\baselineskip}

An ARMA model is stationary when all the roots of it's AR polynomial lie outside the complex unit circle. From inspection, that means models B and D are stationary. We can find the roots of the AR polynomial for model C as $\frac{1.1 \pm \sqrt{1.1^2-4(0.8)(1)}}{2(0.8)}=\frac{1.1 \pm \sqrt{-1.99}}{1.6}=\frac{11}{16} \pm \frac{i\sqrt{1.99}}{1.6}$ and so the complex modulus is $\sqrt{\frac{121}{256}+\frac{199}{256}}>\sqrt{1}=1$, which means model C is also stationary. All told, models B, C, and D are stationary.
\vspace{\baselineskip}

An ARMA model is invertible if all the roots of it's MA polynomial lie outside the complex unit circle. From inspection, that means model B is invertible. We can find the roots of the MA polynomial for model C as $\frac{1.7 \pm \sqrt{1.7^2-4(0.72)(1)}}{2(0.72)}=\frac{1.7 \pm \sqrt{2.89-2.88}}{1.44}=\frac{1.7 \pm 0.1}{1.44}>1$ and so model C is also invertible. We can find the roots of the MA polynomial for model D as $\frac{1.2 \pm \sqrt{1.2^2-4(0.2)(1)}}{2(0.2)}=\frac{1.2 \pm 0.8}{0.4}$; one of the roots lies on the unit circle and so model D is not invertible. All told, models B and C are invertible.

\newpage
\noindent\textbf{\boldmath{b. Express the model as an infinite MA if the process is stationary.}}
\vspace{\baselineskip}

We equate the coefficients of the backshift operator to get the expression. In the case of Model B, we have:

\vspace{-0.5cm}
\begin{align*}
	(1-0.8B)\tilde{Z}_t&=(1-0.5B)a_t\\
	(1-0.8B)\left(1+\psi_1B+\psi_2B^2+\cdots\right)a_t&=(1-0.5B)a_t\\
\end{align*}
\vspace{-1.0cm}

And so:

\vspace{-0.5cm}
\begin{align*}
	&\psi_1B-0.8B=-0.5B \implies \psi_1=0.3\\
	&\psi_2B^2-0.8\psi_1B^2=0B^2 \implies \psi_2=0.24\\
	&\psi_3B^3-0.8\psi_2B^3=0B^3 \implies \psi_3=0.8(0.24)
\end{align*}

Continuing in this fashion, we have: $\psi_{B_k}=\begin{cases}
	0.3, &\text{$k=1$}\\
	0.8(0.3)^{k-1}, &\text{$k>1$}
\end{cases}$
\vspace{\baselineskip}
\vspace{\baselineskip}
\vspace{\baselineskip}


In the case of model C, we have:

\vspace{-0.5cm}
\begin{align*}
	&(1-1.1B+0.8B^2)\tilde{Z}_t=(1-1.7B+0.72B^2)a_t\\
	&(1-1.1B+0.8B^2)\left(1+\psi_1B+\psi_2B^2+\cdots\right)=(1-1.7B+0.72B^2)a_t
\end{align*}
\vspace{-0.5cm}

And so:

\vspace{-0.5cm}
\begin{align*}
	&\psi_1B-1.1B=-1.7B \implies \psi_1=-0.6\\
	&\psi_2B^2-1.1\psi_1B^2+0.8B^2=0.72B^2 \implies \psi_2=0.72+(1.1 \cdot -0.6)-0.8=-0.74\\
	&\psi_3B^3-1.1\psi_2B^3+0.8\psi_1B^3=0B^3 \implies \psi_3=1.1(-0.74)-0.8(-0.6)
\end{align*}
\vspace{-0.5cm}

Continuing in this way, we achieve the recursion $\psi_{C_k}=1.1(\psi_{C_{k-1}})-0.8(\psi_{C_{k-2}})$ for $k\geq 3$ where $\psi_{C_{k_1}}=-0.6$ and $\psi_{C_{k_2}}=-0.74$.
\vspace{\baselineskip}
\vspace{\baselineskip}
\vspace{\baselineskip}



In the case of model D, we have:

\vspace{-0.5cm}
\begin{align*}
	&(1-0.6B)\tilde{Z}_t=(1-1.2B+0.2B^2)a_t\\
	&(1-0.6B)\left(1+\psi_1B+\psi_2B^2+\cdots\right)=(1-1.2B+0.2B^2)a_t
\end{align*}
\vspace{-0.5cm}
 
 And so:
 
 \vspace{-0.5cm}
 \begin{align*}
 	&\psi_1B-0.6B=-1.2B \implies \psi_1=-0.6\\
 	&\psi_2B^2-0.6\psi_1B^2=0.2B^2 \implies \psi_2=0.2+0.6(-0.6)=-0.16\\
 	&\psi_3B^3-0.6\psi_2B^3=0 \implies \psi_3=0.6(-0.16)
 \end{align*}
\vspace{-0.5cm}

Continuing in this way, we see $\psi_{D_k}=0.6^{k-2}(-0.16)$ for $k \geq 3$ while $\psi_{D_1}=-0.6$ and $\psi_{D_2}=-0.16$.



\newpage
\noindent\textbf{\boldmath{c. Express the model as an infinite AR representation if the process is invertible.}}
\vspace{\baselineskip}

We use the same strategy of equating coefficients. In the case of model B, we have:

\vspace{-0.5cm}
\begin{align*}
	&(1-0.8B)\tilde{Z}_t=(1-0.5B)a_t\\
	&(1-0.8B)\tilde{Z}_t=(1-0.5B)(1+\pi_1B+\pi_2B^2+\cdots)\tilde{Z}_t
\end{align*}
\vspace{-0.5cm}

And so:

\vspace{-0.5cm}
\begin{align*}
	&-0.8B=\pi_1B-0.5B \implies \pi_1=-0.3B\\
	&0=\pi_2B^2-0.5\pi_1B^2 \implies \pi_2=(0.5)(-0.3)
\end{align*}
\vspace{-0.5cm}

Continuing in this way, we have $\pi_{B_1}=-0.3$ and $\pi_{B_k}=0.5\pi_{B_{k-1}}$ for $k \geq 2$.
\vspace{\baselineskip}
\vspace{\baselineskip}
\vspace{\baselineskip}



In the case of model C, we have:

\vspace{-0.5cm}
\begin{align*}
	&(1-1.1B+0.8B^2)\tilde{Z}_t=(1-1.7B+0.72B^2)a_t\\
	&(1-1.1B+0.8B^2)\tilde{Z}_t=(1-1.7B+0.72B^2)(1+\pi_1B+\pi_2B^2+\cdots)\tilde{Z}_t
\end{align*}
\vspace{-0.5cm}

And so:
\vspace{-0.5cm}
\begin{align*}
	&-1.1B=\pi_1B-1.7B \implies \pi_1=0.6\\
	&0.8B^2=\pi_2B^2-1.7\pi_1B^2 +0.72B^2 \implies \pi_2=0.8-0.72+1.7(0.6)=1.1\\
	&0=\pi_3B^3-1.7\pi_2B^3++0.72\pi_1B^3 \implies \pi_3=1.7(1.1)+0.72(0.6)=2.302
\end{align*}
\vspace{-0.5cm}

Continuing in this fashion, we see $\pi_{C_k}=\begin{cases}
	0.6, &\text{$k=1$}\\
	1.1, &\text{$k=2$}\\
	1.7\pi_{C_{k-1}}-0.72\pi_{C_{k-2}}, &\text{$k \geq 3$}
\end{cases}$




\newpage
\noindent\textbf{\boldmath{2) Consider the model $\tilde{Z}_t=0.3\tilde{Z}_{t-1}+0.34\tilde{Z}_{t-2}-0.12\tilde{Z}_{t-3}+a_t-0.7a_{t-1}+0.12a_{t-2}$. Determine whether or not the model is in reduced form, and if it is not, find the reduced form.}}
\vspace{\baselineskip}

The model is in reduced form if the AR and MA polynomials share no common factors. We go directly for the definition. The model is $(1-0.3B-0.34B^2+0.12B^3)\tilde{Z}_t=(1-0.7B+0.12B^2)a_t$. The quadratic factors as $(1-0.4B)(1-0.3B)$. Dividing the first factor into the cubic, we see $(0.12B^3-0.34B^2-0.3B+1)=(1-0.4B)(-0.3B^2+0.1B+1)$. Then we can write our model as $(-0.3B^2+0.1B+1)\widetilde{Z}_t=(1-0.3B)a_t$.




\newpage
\noindent\textbf{\boldmath{3) Consider the model $(1-B)^2Z_t=(1-0.3B-0.5B^2)a_t$.}}

\vspace{\baselineskip}
\noindent\textbf{\boldmath{a. Is the model for $Z_t$ a stationary model? Why or why not?}}
\vspace{\baselineskip}

$Z_t$ is not stationary since it's AR model has two roots the lie \textit{on} the complex unit circle.



\vspace{\baselineskip}
\noindent\textbf{\boldmath{b. Is the model for $W_t=(1-B)^2Z_t$ a stationary model? Why or why not?}}
\vspace{\baselineskip}

$W_t$ is stationary since it is a finite MA model, and finite MA's are always stationary.


\vspace{\baselineskip}
\noindent\textbf{\boldmath{c. Determine the autocorrelation function for $W_t$.}}
\vspace{\baselineskip}

The autocovariance function is $\E(W_{t}W_{t-k})$:

\vspace{-0.5cm}
\begin{align*}
	\E(W_tW_{t-k})=&\E\left((a_t-0.3a_{t-1}-0.5a_{t-2})(a_{t-k}-0.3a_{t-k-1}-0.5a_{t-k-2})\right)\\
	=&\E(a_ta_{t-k}) -0.3\E(a_ta_{t-k-1})-0.5\E(a_ta_{t-k-2})\\
	&-0.3\E(a_{t-1}a_{t-k})+0.09\E(a_{t-1}a_{t-k-1})+0.15\E(a_{t-1}a_{t-k-2})\\
	&-0.5\E(a_{t-2}a_{t-k})+0.15\E(a_{t-2}a_{t-k-1})+0.25\E(a_{t-2}a_{t-k-2})
\end{align*}

By the properties of white noise, $\E(a_{t}a_{t-k})=\begin{cases}
	\sigma_a^2, &\text{$k=0$}\\
	0, &\text{$k>0$}
\end{cases}$. So:

\begin{align*}
	\gamma_0&=\sigma_a^2(1+0.09+0.25)=1.34\sigma_a^2\\ \gamma_1&=\sigma_a^2(-0.3+0.15)=-0.15\sigma_a^2\\
	\gamma_2&=\sigma_a^2(-0.5)=-0.5\sigma_a^2\\
	\gamma_3&=\sigma_a^2()\\
	\gamma_4&=0
\end{align*} . Dividing by $\gamma_0$ gives us the correlation function (here $s>2$):

\begin{align*}
	\rho_0&=1\\
	\rho_1&=\frac{-0.15}{1.34}\approx -0.11\\
	\rho_2&=\frac{-0.5}{1.34}\approx -0.37\\
	\rho_{s}&=0
\end{align*}


\end{document}